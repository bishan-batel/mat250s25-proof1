\documentclass{exam}

\usepackage{amsfonts}
\usepackage{amssymb}
\usepackage{mathtools}
\usepackage{braket}
\usepackage{forloop}
\usepackage{amsthm}
\usepackage[backend=biber]{biblatex}

\theoremstyle{plain}
\newtheorem{assumption}{Assumption}

\theoremstyle{definition}
\newtheorem{definition}{Definition}

\newtheorem{lemma}{Lemma}

\input{preamble.sty}

\addbibresource{references.bib}

\begin{document}

\title{MAT250S25 Proof 1}
\author{Kishan S Patel}
\maketitle

% \noindent\rule{\textwidth}{1pt}

% \noindent\rule{\textwidth}{1pt}

\noindent \rule{\textwidth}{1pt}
\linebreak

\begin{thereom}
	Any set $B$ is Linearly Independent if it is a subset of a linearly independent set.

	$$B \in L_I \iff  (B \subset A)  \wedge A \in L_I $$

	\label{subset}
\end{thereom}
\noindent \rule{\textwidth}{1pt}
\linebreak

Let $\R^n$ be any Euclidian Space, and let $\vec v_1, \vec v_2, \vec v_3 \in \R^n$ be vectors such that the set
of vectors $\set{\vec v_1, \vec v_2, \vec v_3}$ is linearly independent.


Prove that the set of vectors $\set{\vec v_1, \vec v_2, \vec v_1 + \vec v_3}$ is linearly independent (Proposition $P$).

\renewcommand\qedsymbol{QED}

\begin{proof}

	Let $Q$ be the proposition that $\set{\vec v_1, \vec v_2, \vec v_3}$ is linearly independent.

	Assume that $Q \wedge \neg P \rightarrow \set{\vec v_1, \vec v_2, \vec v_1 + \vec v_3}$ is linearly dependent.

	Let $\mathcal F$ be the set of all sets that have elements $\in \R$, but not any set with all 0 values.

	$$
		\neg P \rightarrow \exists \set{x_1,x_2,x_3} \in \mathcal F : x_1 \vec v_1 + x_2 \vec v_2 + x_3 \pa{\vec v_1 + \vec v_3} = \vec 0
	$$
	\begin{align*}
		\sim
		\pa{x_1 + x_3} {\vec v_1} + x_2 \vec v_2 + x_3 \vec v_3    & = \vec 0         \\
		\pa{x_1+x_3}\vec v_1 + x_2\vec v_2                         & = - x_3 \vec v_3 \\
		-\pa{\frac{x_1}{x_3} + 1}\vec v_1 -\frac{x_2}{x_3}\vec v_2 & = \vec v_3
	\end{align*}

	$$
		\llet t = -\frac{x_1}{x_3}, s=-\frac{x_2}{x_3}
	$$

	\begin{align*}
		\neg P \implies \exists t,v : t\vec v_1 + s \vec v_2 & = \vec v_3
	\end{align*}

	Assuming $\neg P$ implies that there exists two real numbers $\set{t, v}\in \mathcal{F}$  that can be made into a linear combination with $\set{\vec v_1, \vec v_2}$ to construct $\vec v_3$.
	Therefore the set $\set{\vec v_1, \vec v_2, \vec v_3}$ is linearly dependent.

	As shown by thereom \ref{subset}, $\set{\vec v_1, \vec v_2}$ must be linearly independent as it is a subset of the defined linearly independent set $\set{\vec v_1, \vec v_2, \vec v_3}$.

	This shows that $Q\wedge \neg P \implies \neg Q$, which is a contradiction, meaning that $P$ must be true ($\set{\vec v_1,\vec v_2, \vec v_1+\vec v_3}$ is linearly independent).

	% Let $\set{x_1,x_2,x_3} \in \mathcal F$ such that
	% $x_1\vec v_1 + x_2 \vec v_2 + x_3\pa{\vec v_1+\vec v_3} =\vec 0.$


\end{proof}


\end{document}

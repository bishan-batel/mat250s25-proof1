\documentclass{exam}

\usepackage{amsfonts}
\usepackage{amssymb}
\usepackage{mathtools}
\usepackage{braket}
\usepackage{forloop}
\usepackage{amsthm}
\usepackage[backend=biber]{biblatex}

\theoremstyle{plain}
\newtheorem{assumption}{Assumption}

\theoremstyle{definition}
\newtheorem{definition}{Definition}

\newtheorem{lemma}{Lemma}


% huge align
\newcommand{\ha}[1]{{\huge{\begin{align*}#1\end{align*}}}}



% P & S Are excluded 
% \newcommand{\A}[0]{{\mathbb A}}
% \newcommand{\B}[0]{{\mathbb B}}
% \newcommand{\C}[0]{{\mathbb C}}
% \newcommand{\D}[0]{{\mathbb D}} 
% \newcommand{\E}[0]{{\mathbb E}} 
% \newcommand{\F}[0]{{\mathbb F}} 
% \newcommand{\G}[0]{{\mathbb G}} 
% \newcommand{\H}[0]{{\mathbb H}} 
% \newcommand{\I}[0]{{\mathbb I}} 
% \newcommand{\J}[0]{{\mathbb J}} 
% \newcommand{\K}[0]{{\mathbb K}} 
% \newcommand{\L}[0]{{\mathbb L}} 
% \newcommand{\M}[0]{{\mathbb M}} 
\newcommand{\N}[0]{{\mathbb N}}
% \newcommand{\O}[0]{{\mathbb O}} 
% \newcommand{\P}[0]{{\mathbb P}} 
\newcommand{\Q}[0]{{\mathbb Q}}
\newcommand{\R}[0]{{\mathbb R}}
% \renewcommand{\S}[0]{{\mathbb S}}
% \newcommand{\T}[0]{{\mathbb T}}
% \newcommand{\U}[0]{{\mathbb U}}
% \newcommand{\V}[0]{{\mathbb V}}
% \newcommand{\W}[0]{{\mathbb W}}
% \newcommand{\X}[0]{{\mathbb X}}
% \newcommand{\Y}[0]{{\mathbb Y}}
\newcommand{\Z}[0]{{\mathbb Z}}

% Calculus
% \newcommand{\d}[0]{{\mathrm{d}}}
\newcommand{\deriv}[2]{ \frac{ \d{#1} }{ \d{#2} } }
\newcommand{\pderiv}[2]{ \frac{ \partial{#1} }{ \partial{#2} } }

\newcommand{\nderiv}[3]{ \frac{ \d^{#1}{#2} }{ \d{#3}^{#1} } }
\newcommand{\npderiv}[3]{ \frac{ \partial^{#1}{#2} }{ \partial{#3}^{#1} } }

% Linear Algebra 
\renewcommand{\vector}[1]{ \overrightarrow{#1} }
\newcommand{\vecn}[1]{ {\hat #1} }

\newcommand{\mat}[1]{{ \begin{bmatrix} #1 \end{bmatrix} }}
\newcommand{\mats}[1]{{ \ba{\begin{smallmatrix} #1 \end{smallmatrix}} }}

\newcommand{\pmat}[1]{{ \begin{pmatrix} #1 \end{pmatrix} }}
\newcommand{\pmats}[1]{{ \pa{\begin{smallmatrix} #1 \end{smallmatrix}} }}

\newcommand{\emat}[1]{{ \begin{ematrix} #1 \end{ematrix} }}
\newcommand{\emats}[1]{{ \begin{smallmatrix} #1 \end{smallmatrix} }}

\newcommand{\vmat}[1]{{ \begin{vmatrix} #1 \end{vmatrix} }}

\newcommand{\rowechelon}[1]{{
			\left[\begin{array}{ccc|c} #1 \end{array}\right]
		}}

\newcommand{\augmented}[2]{{
			\left[\begin{array}{#1} #2 \end{array}\right]
		}}

% Generic Notatino
\newcommand{\paren}[1]{{ \left(#1\right) }}

\newcommand{\pa}[1]{{ \left(#1\right) }}
\newcommand{\ba}[1]{{ \left[#1\right] }}


\newcommand{\llet}[0]{ {\text{let } } }
\newcommand{\undefined}[0]{ {\text{undefined.} } }

\newcommand{\op}[1]{ {\operatorname{#1} } }

\newcommand{\brt}[2]{ {\root {#1} \of {#2} } }

\newcommand{\proj}[1]{ { \op{proj}_{#1} }}
\newcommand{\projperp}[1]{ { \op{proj}_{#1\perp} } }

\newcommand{\norm}[1]{{ {\left\lVert #1 \right\rVert} }}
\newcommand{\norms}[1]{{ {\lVert #1 \rVert} }}

% CS 
\newcommand{\hex}[1]{{ \pa{\mathrm{#1}}_{16} }}
\newcommand{\bin}[1]{{ \pa{#1}_{2} }}
\newcommand{\binb}[2]{{ \pa{#1}^{#2}_{2} }}
\newcommand{\dec}[1]{{ \pa{#1}_{10} }}


\newcommand{\true}[0]{{ \mathrm{true} }}
\newcommand{\false}[0]{{ \mathrm{false} }}

\renewcommand{\ba}[1]{{ \left[ {#1} \right] }}

\newcommand{\ceil}[1]{{ \left\lceil {#1} \right\rceil }}
\newcommand{\floor}[1]{{ \left\lfloor {#1} \right\rfloor }}

\newcommand{\ang}[1]{{ \left\langle {#1} \right\rangle }}

\newcommand{\transpose}[1]{ { {#1}^{\intercal} } }





\addbibresource{references.bib}

\begin{document}

\title{MAT250S25 Proof 1}
\author{Kishan S Patel}
\maketitle

% \noindent\rule{\textwidth}{1pt}

% \noindent\rule{\textwidth}{1pt}

\noindent \rule{\textwidth}{1pt}
\linebreak

\begin{thereom}
	Any set $B$ is Linearly Independent if it is a subset of a linearly independent set.

	$$B \in L_I \iff  (B \subset A)  \wedge A \in L_I $$

	\label{subset}
\end{thereom}
\noindent \rule{\textwidth}{1pt}
\linebreak

Let $\R^n$ be any Euclidian Space, and let $\vec v_1, \vec v_2, \vec v_3 \in \R^n$ be vectors such that the set
of vectors $\set{\vec v_1, \vec v_2, \vec v_3}$ is linearly independent.


Prove that the set of vectors $\set{\vec v_1, \vec v_2, \vec v_1 + \vec v_3}$ is linearly independent (Proposition $P$).

\renewcommand\qedsymbol{QED}

\begin{proof}

	Let $Q$ be the proposition that $\set{\vec v_1, \vec v_2, \vec v_3}$ is linearly independent.

	Assume that $Q \wedge \neg P \rightarrow \set{\vec v_1, \vec v_2, \vec v_1 + \vec v_3}$ is linearly dependent.

	Let $\mathcal F$ be the set of all sets that have elements $\in \R$, but not any set with all 0 values.

	$$
		\neg P \rightarrow \exists \set{x_1,x_2,x_3} \in \mathcal F : x_1 \vec v_1 + x_2 \vec v_2 + x_3 \pa{\vec v_1 + \vec v_3} = \vec 0
	$$
	\begin{align*}
		\sim
		\pa{x_1 + x_3} {\vec v_1} + x_2 \vec v_2 + x_3 \vec v_3    & = \vec 0         \\
		\pa{x_1+x_3}\vec v_1 + x_2\vec v_2                         & = - x_3 \vec v_3 \\
		-\pa{\frac{x_1}{x_3} + 1}\vec v_1 -\frac{x_2}{x_3}\vec v_2 & = \vec v_3
	\end{align*}

	$$
		\llet t = -\frac{x_1}{x_3}, s=-\frac{x_2}{x_3}
	$$

	\begin{align*}
		\neg P \implies \exists t,v : t\vec v_1 + s \vec v_2 & = \vec v_3
	\end{align*}

	Assuming $\neg P$ implies that there exists two real numbers $\set{t, v}\in \mathcal{F}$  that can be made into a linear combination with $\set{\vec v_1, \vec v_2}$ to construct $\vec v_3$.
	Therefore the set $\set{\vec v_1, \vec v_2, \vec v_3}$ is linearly dependent.

	As shown by thereom \ref{subset}, $\set{\vec v_1, \vec v_2}$ must be linearly independent as it is a subset of the defined linearly independent set $\set{\vec v_1, \vec v_2, \vec v_3}$.

	This shows that $Q\wedge \neg P \implies \neg Q$, which is a contradiction, meaning that $P$ must be true ($\set{\vec v_1,\vec v_2, \vec v_1+\vec v_3}$ is linearly independent).

	% Let $\set{x_1,x_2,x_3} \in \mathcal F$ such that
	% $x_1\vec v_1 + x_2 \vec v_2 + x_3\pa{\vec v_1+\vec v_3} =\vec 0.$


\end{proof}


\end{document}

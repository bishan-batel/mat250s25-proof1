\begin{proof}

	Let $\mathcal F$ be the set of all ordered sets that have elements $\in \R$, but not any ordered set with all 0 values.

	Let $Q$ be the proposition that $\set{\vec v_1, \vec v_2, \vec v_3}$ is linearly independent.

	Assuming $Q \wedge \neg P \rightarrow \set{\vec v_1, \vec v_2, \vec v_1 + \vec v_3}$ is linearly dependent.

	$$
		\neg P \rightarrow \exists \set{x_1,x_2,x_3} \in \mathcal F : x_1 \vec v_1 + x_2 \vec v_2 + x_3 \pa{\vec v_1 + \vec v_3} = \vec 0
	$$
	\begin{align*}
		\sim
		\pa{x_1 + x_3} {\vec v_1} + x_2 \vec v_2 + x_3 \vec v_3    & = \vec 0         \\
		\pa{x_1+x_3}\vec v_1 + x_2\vec v_2                         & = - x_3 \vec v_3 \\
		-\pa{\frac{x_1}{x_3} + 1}\vec v_1 -\frac{x_2}{x_3}\vec v_2 & = \vec v_3
	\end{align*}

	$$
		\llet t = -\frac{x_1}{x_3}, s=-\frac{x_2}{x_3}
	$$

	\begin{align*}
		\neg P \implies \exists t,v : t\vec v_1 + s \vec v_2 & = \vec v_3
	\end{align*}

	This brings a contradiction: assuming $\neg P$ implies that there exists two real numbers $t, v$ (who are not both $0$) that can be made into a linear combination with $\set{\vec v_1, \vec v_2}$ to equal $\vec v_3$, therefore $\neg P \implies \set{\vec v_1, \vec v_2, \vec v_3}$ is linearly dependent.

	As shown by Lemma \ref{subset}, $\set{\vec v_1, \vec v_2}$ is linearly independent as it is a subset of the linearly independent set~$\set{\vec v_1, \vec v_2, \vec v_3}$.

	This shows that $Q\wedge \neg P \implies \neg Q$, which is a contradiction which must mean that $P$ must be true, and the set $\set{\vec v_1,\vec v_2, \vec v_1+\vec v_3}$ is linearly independent.

	% Let $\set{x_1,x_2,x_3} \in \mathcal F$ such that
	% $x_1\vec v_1 + x_2 \vec v_2 + x_3\pa{\vec v_1+\vec v_3} =\vec 0.$


\end{proof}
